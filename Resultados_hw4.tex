\documentclass[twoside]{article}

\usepackage{lipsum} % Package to generate dummy text throughout this template

\usepackage[sc]{mathpazo} % Use the Palatino font
\usepackage[T1]{fontenc} % Use 8-bit encoding that has 256 glyphs
\linespread{1.05} % Line spacing - Palatino needs more space between lines
\usepackage{microtype} % Slightly tweak font spacing for aesthetics

\usepackage[hmarginratio=1:1,top=32mm,columnsep=20pt]{geometry} % Document margins
\usepackage{multicol} % Used for the two-column layout of the document
\usepackage[hang, small,labelfont=bf,up,textfont=it,up]{caption} % Custom captions under/above floats in tables or figures
\usepackage{booktabs} % Horizontal rules in tables
\usepackage{float} % Required for tables and figures in the multi-column environment - they need to be placed in specific locations with the [H] (e.g. \begin{table}[H])
\usepackage{hyperref} % For hyperlinks in the PDF

\usepackage{lettrine} % The lettrine is the first enlarged letter at the beginning of the text
\usepackage{paralist} % Used for the compactitem environment which makes bullet points with less space between them
\usepackage[utf8]{inputenc}

\usepackage{abstract} % Allows abstract customization
\renewcommand{\abstractnamefont}{\normalfont\bfseries} % Set the "Abstract" text to bold
\renewcommand{\abstracttextfont}{\normalfont\small\itshape} % Set the abstract itself to small italic text

\usepackage{titlesec} % Allows customization of titles
\renewcommand\thesection{\Roman{section}} % Roman numerals for the sections
\renewcommand\thesubsection{\Roman{subsection}} % Roman numerals for subsections
\titleformat{\section}[block]{\large\scshape\centering}{\thesection.}{1em}{} % Change the look of the section titles
\titleformat{\subsection}[block]{\large}{\thesubsection.}{1em}{} % Change the look of the section titles
\usepackage[normalem]{ulem}

\usepackage{graphicx}
\graphicspath{ {images/} }
\usepackage{fancyhdr} % Headers and footers
\pagestyle{fancy} % All pages have headers and footers
\fancyhead{} % Blank out the default header
\fancyfoot{} % Blank out the default footer
\fancyhead[C]{Abril 2017 $\bullet$ Universidad de los Andes $\bullet$   Métodos Computacionales} % Custom header text
\fancyfoot[RO,LE]{\thepage} % Custom footer text
\usepackage{textcomp}
\usepackage{siunitx}


%-------------------------------------------------
%\usepackage{eso-pic}
%\usepackage[demo]{graphicx}
%\newcommand\AtPageUpperRight[1]{\AtPageUpperLeft{%
   %\makebox[\paperwidth][r]{#1}}}
   
%\AddToShipoutPictureBG*{%
 % \AtPageUpperRight{\raisebox{-\height}{\includegraphics[scale=0.5]{andes}}}}
 


%---------------------------------------
%	TITLE SECTION
%----------------------------------------------------------------------------------------

\title{\vspace{-15mm}\fontsize{20pt}{12pt}\selectfont\textbf{ Ecuaciones diferenciales parciales \\ Ecuación de difusión en 2 dimensiones \\ }} 

\author{
  Sebastián Del Río Navarro\\
  \texttt{201417736}\\
  }
 
%----------------------------------------------------------------------------------------

\begin{document}
\maketitle % Insert title
\thispagestyle{fancy} % All pages have headers and footers

%----------------------------------------------------------------------------------------
%	ABSTRACT
%----------------------------------------------------------------------------------------


%----------------------------------------------------------------------------------------
%	ARTICLE CONTENTS
%----------------------------------------------------------------------------------------


\section{Introducción}

La ecuación de difusión de temperatura esta dada por: 

\begin{equation}
    \frac{\partial T(t,x,y)}{\partial t} = v \frac{\partial ^2 T(t,x,y)}{\partial ^2 x} + v \frac{\partial ^2 T(t,x,y)}{\partial ^2 y}
\end{equation}

Para el ejercicio de una placa cuadrada de 1m de lado, su coeficiente de difusión es $v$ = $10 ^{-4}$. Se resolvió para 3 diferentes tipos de condiciones de frontera. 

1. Abiertas: Las condiciones iniciales son tales que toda la placa está a T = 50°C excepto por un pequeñp rectángulo de 20cm X 10cm que está a T = 100°C y está localizado a x = 20cm del lado izquierdo de la placa y centrado en y. 

2. Periódicas: El mismo pequeño rectángulo tiene una fuente de calor que mantiene su temperatura constante a T = 100°C. 

3. Fijas: T = 50°C. 

Para resolver la ecuación fue necesario su discretización, con lo cual se obtuvo lo siguiente: 

\begin{equation}
    T_{i,j}^{k+1}  = \lambda \upsilon [T_{i+1,j} + T_{i-1,j} + T_{i,j+1} + T_{i,j-1} - 4 T_{i,j}^{k}] + T_{i,j}^{k} 
\end{equation}




%------------------------------------------------

\section{Condiciones de Frontera Abiertas}

En estas condiciones de frontera, la energía térmica puede ser transferida desde la placa a un entorno por medio de sus límites. 


%------------------------------------------------

\section{Condiciones de frontera Periódicas}

En estas condiciones de frontera se asocian los estados de los puntos de las fronteras de los lados con los lados opuestos y las bases con las bases opuestas de la placa. 


%------------------------------------------------

\section{Condiciones de frontera Fijas}

En estas condiciones de frontera, las fronteras de la placa se mantienen a una temperatura constante. 

Grafica para el promedio de la temperatura en el tiempo 
\includegraphics[scale=0.6]{FIJAS.png}
\captionof{figure}{ Tomado de Clasificación de imágenes multiespectrales ASTER mediante funciones adaptativas.
Quiron (2009)}

%-----------------------------------------------

\section{Referencias}




\end{document}
